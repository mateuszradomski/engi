\chapter{Wprowadzenie}
\label{cha:wprowadzenie}

Procesory graficzne z wielu punktów widzenia są kompletnie różne od zwykłej jednostki centralnej w każdym komputerze.
Mają nienaturalną konstrukcję, skupiającą się na posiadaniu jak największej ilości jednostek arytmetycznych w krzemie.
Wynikiem tego jest nieosiągalnie duża dla zwykłych mikroprocesorów surowa moc obliczeniowa.
Niecodzienność struktury tej rodziny procesorów czyni tworzenie programów rozwiązujących dany problem bardziej skomplikowane.
\emph{Spektrum kodu, który zostanie zaakceptowany przez kartę graficzną jest szerokie w porównaniu do zwykłych mikroprocesorów.
Normalnie procesor jest w stanie przyjąć jedynie instrukcje opisane i wspierane z danej standardowej architektury przemysłowej (ISA).
}
Procesory graficzne mogą wspierać różne abstrakcje, które pomimo różnic udostępniają te same podstawowe operacje.
Umożliwia to ciągle działający sterownik, który tłumaczy daną abstrakcję na odpowiedni dla tej mikro-architektury kod maszynowy.
Taką abstrakcją jest język pośredni SPIR-V stworzony na potrzeby obliczeń wielowątkowych i graficznych.
Załadowanie i uruchomienie tego kodu umożliwia specyfikacja Vulkan, opisująca zestaw interfejsów pozwalających na kontrolowanie zachowania procesora graficznego.

%---------------------------------------------------------------------------

\section{Cele pracy}
\label{sec:celePracy}

Celem poniższej pracy jest implementacja algorytmów mnożenia macierzy rzadkich przez wektor oraz przedstawienie użytego interfejsu do procesora graficznego. Praca również omawia ogólną mikro-architekturę procesorów graficznych i uruchamianie shaderów obliczeniowych na nich.

%---------------------------------------------------------------------------

\section{Zawartość pracy}
\label{sec:zawartoscPracy}

W rozdziale \ref{cha:macierzwektor} omówiono problem macierzy rzadkich i formatów do ich przetrzymywania.
Następnie w rozdziale \ref{cha:vulkan_micro_shaders} przedstawiono ogólną mikro-architekturę nowoczesnych procesorów graficznych.
Opisano użyty interfejs Vulkan do zarządzania procesorem graficznym oraz czym są shadery obliczeniowe.
Sposób implementacji komunikacji z procesorem graficznych oraz poszczególnych shaderów obliczeniowych zawarto w rozdziale \ref{cha:implementacja}.
Uzyskane wyniki używające tej implementacji przeanalizowano w rozdziale \ref{cha:wyniki}, a podsumowanie pracy znajduje się w rozdziale \ref{cha:podsumowanie}.















