\chapter{Wprowadzenie}
\label{cha:wprowadzenie}

Procesory graficzne z wielu punktów widzenia są kompletnie różne od zwykłej jednostki centralnej w każdym komputerze.
Mają nienaturalną konstrukcję, skupiającą się na posiadaniu jak największej ilości jednostek arytmetycznych w krzemie.
Wynikiem tego jest nieosiągalnie duża dla zwykłych mikroprocesorów, surowa moc obliczeniowa.
Niecodzienność struktury tej rodziny procesorów czyni tworzenie programów rozwiązujących dany problem bardziej skompilowane.
Ilość dostępnych języków programowania dla procesorów graficznych jest znikoma w porównaniu do języków pozwalających na programowanie przykładowo jednostki centralnej z architekturą x86.
Procesory graficzne mogą wspierać różne ISA w tej samej chwili, umożliwia to ciągle działający sterownik, który tłumaczy daną abstrakcję na odpowiedni dla tej mikro-architektury kod maszynowy.
Tak jak w nowoczesnych jednostkach centralnych, finalnie wykonywany kod maszynowy nazywany mikrokodem, często zmienia się pomiędzy architekturami; dla przykładu firma AMD udostępnia specyfikację mikrokodu publicznie\cite{AMDISA}, natomiast firma NVIDIA nie udostępnia takich danych. 
Przykładową abstrakcją wyższą jest język pośredni SPIR-V stworzony na potrzeby obliczeń wielowątkowych i graficznych.
Załadowanie i uruchomienie tego kodu umożliwia specyfikacja Vulkan, opisująca zestaw interfejsów pozwalających na kontrolowanie zachowania procesora graficznego.

%---------------------------------------------------------------------------

\section{Cele pracy}
\label{sec:celePracy}

Celem poniższej pracy jest implementacja algorytmów mnożenia macierzy rzadkich przez wektor oraz przedstawienie użytego interfejsu do procesora graficznego. Praca również omawia ogólną mikro-architekturę procesorów graficznych i uruchamianie shaderów obliczeniowych na nich.

%---------------------------------------------------------------------------

\section{Zawartość pracy}
\label{sec:zawartoscPracy}

W rozdziale \ref{cha:macierzwektor} omówiono problem macierzy rzadkich i formatów do ich przetrzymywania.
Następnie w rozdziale \ref{cha:vulkan_micro_shaders} przedstawiono ogólną mikro-architekturę nowoczesnych procesorów graficznych.
Opisano użyty interfejs Vulkan do zarządzania procesorem graficznym, czym są shadery obliczeniowe.
Sposób implementacji komunikacji z procesorem graficznych oraz poszczególnych shaderów obliczeniowych zawarto w rozdziale \ref{cha:implementacja}.
Uzyskane wyniki używając tej implementacji przeanalizowano w rozdziale \ref{cha:wyniki}, a podsumowanie pracy przedstawiono w rozdziale \ref{cha:podsumowanie}.















